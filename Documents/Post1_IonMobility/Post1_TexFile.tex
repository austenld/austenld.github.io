\documentclass[10pt,a4paper,final]{article}
\usepackage[latin1]{inputenc}
\usepackage{amsmath}
\usepackage{amsfonts}
\usepackage{amssymb}
\usepackage{graphicx}

%%%%%%%%%%%%%%%%%%%%%
%---
%layout: post
%title: Introduction IMS Theory
%---

\begin{document}
\section{Introduction IMS Theory}
This is an introduction to the basics of ion mobility spectrometry (IMS). The purpose of this page is to introduce you to the concept of IMS and how ions behave in neutral gasses. The structure and content of this page is based upon lecture slides written by Dr. Bill Siems.
\section{Variables in IMS}
Here we introduce the commonly referenced variables which will be discussed in further detail. These variables are grouped into three categories: instrument-state, mass, and measured quantities.
\subsection{Instrument-State Variables: $(L, V, E)$ and $(N, T, P)$}
$L$ is the length of the drift space and $V$ is the voltage drop across the drift space. The electric field for a traditional linear drift tube is constant and equivalent to $E=\frac{V}{L}$. $N$ is the number density of the drift gas. If the drift gas is considered ideal then $N=N_0\cdot \dfrac{P\cdot T_0}{P_0\cdot T}$, where $T_0=273.15\;$K, $P_0= 760\;$torr, and  $N_0=2.687\cdot10^{19} \text{cm}^3$ (the number density of an ideal gas at STP).
\subsection{Mass Parameters: $(m, M, q=Ze)$}
$m$ and $M$ correspond to the ion and drift gas masses respectively and $q$ corresponds to the charge of the ion.
\subsection{Measured Quantities: $(t_d, v_d)$}
$t_d$ is the drift time required for an ion swarm to traverse the length of the drift space $L$. This terminal velocity is called the drift velocity $(v_d)$. The drift velocity of a given ion population is equivalent to $v_d=\dfrac{L}{t_d}$ where the length is often predetermined and the drift time is measured.
\section{Ion Mobility ($K$)}
The mobility $(K)$ of an ion swarm is important and related to the drift velocity of an ion where $K=\frac{v_d}{E}$.
The central observation is that if an electric field $(E)$ is applied to ions dispersed in gas, the ions move with a characteristic average terminal velocity $(v_d)$ in the direction of the field. The mobility $K$ is defined as the ratio of the velocity to the field strength and is traditionally reported in units of cm$^2$V$^{-1}$sec$^{-1}$.
\section{Five Underlying Assumptions of IMS}
\begin{enumerate}
	\item $n << N$: Ions have a much lower number density then neutrals (additionally, mutual coulombic repulsion of ions is unimportant, ion-ion collisions unimpportant, and each neutral encounters 0 or 1 ion during a mobility experiment).
	\item Collisions are instantaneous.
	\item Three-body collisions are rare.
	\item Ions reach a terminal velocity defined by $v_d=KE$ (vacuum and  ''low'' pressures excluded from consideration).
	\item Ion-neutral reactions and clustering may be ignored (in real experiments clustering is common).	
\end{enumerate}
\section{Key IMS Scaling Factor ($E/N$- Reduced Field Strength)}
$E/N$ is known as the reduced field strength. Experimentally, $K$ is understood to be constant at low reduced field strengths. We shall now see why this parameter is useful by observing from this simulation the effect of varying the reduced field strength $(E/N)$.
\subsection{No Electric Field ($E/N=0$)}
First, we can consider the effect when ions are confined in a box in the presence of neutrals without the presence of an electric field.
%<video width="750" height="375" controls="controls">)
%<source src="/animations/IMS_Theory/NoEfield.mp4" type="video/mp4">)
%</video>
The red squares represent the ions and the blue dots correspond to the neutrals present. Without the presence of an electric field, the ions travel in a pattern of motion known as Brownian motion. The ions behave similarly to that of the neutral gasses.

\subsection{Constant Electric Field}
Now the motion of the ions change when a constant electric field is applied. The voltage decreases linearly in the direction from left to right.
	%<video width="750" height="375" controls="controls" align="center">)
	%<source src="/animations/IMS_Theory/Efield1xN1x.mp4" type="video/mp4">)
	%</video>

\subsection{Double Electric Field}
The ions travel with a higher terminal velocity $(v_d)$ when the electric field is doubled proportional to the defined mobility $(K)$ of the ion population.


\subsection{Double Electric Field and Number Density}
The reduced field strength remains constant when the number density and electric field are doubled. The animation on the right highlights that the terminal velocity $(v_d)$ of the ion population is similar in the two animations
	%When the number density is doubled without changing the electric field setting from previous, the ions travel slower
	%<video width="750" height="375" controls="controls" align="center">)
	%<source src="/animations/IMS_Theory/Efield1xN2x.mp4" type="video/mp4">)
%	</video>
	
%	The electric field setting is now doubled.
%	<video width="750" height="375" controls="controls" align="center">)
%	<source src="/animations/IMS_Theory/Efield2xN2x.mp4" type="video/mp4">)
%	</video>
	
	
	
\end{document}