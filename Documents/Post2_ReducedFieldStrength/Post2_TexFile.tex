\documentclass[10pt,a4paper,final]{article}
\usepackage[latin1]{inputenc}
\usepackage{amsmath}
\usepackage{amsfonts}
\usepackage{amssymb}
\usepackage{graphicx}
\title{Electric field and number density}

%%%%%%%%%%%%%%%%%%%%%
%---
%layout: post
%title: Introduction IMS Theory
%---

\begin{document}
The first part demonstrates why the drift velocity of an ion population is dependent on the applied electric field and number density in context of IMS. The second part examines the key scaling factor known as field strength and its relation to drift velocity.

%Include E1N1 here?

\section{Electric Field $(E)$ and Number Density $(N)$}

%Include animation here

The drift velocity of an ion population $(v_d)$ is dependent on the applied electric field $(E)$ and number density $(N)$. 
The electric field $(E)$ for a traditional linear drift tube is constant and the number density $(N)$ of the drift gas is determined (under ideal conditions) by the ideal gas law ($PV=nRT=Nk_bT$).

The animations below is split into four quadrants where positively charged ions (red diamonds) and neutrals (blue circles) are confined in the box where only elastic collisions are hereby considered. The ions are initially placed in the center x-axis with a starting initial velocity of zero. The neutrals are dispersed in the confined box with random starting initial velocities and positions.

%Animation goes here
In the animation above, the first and second column corresponds to a simulation in the presence of no electric field and applied constant electric field (the direction of the field is from left to right ) respectively. Similarly, the first row corresponds to where the number density is effectively reduced to zero (mimicking high vacuum conditions) and the second row includes the presence of neutrals and thereby increase the number density of gas molecules.


\subsection{(top-left) no electric field and reduced number density}
First, we can consider the simple example when ions are confined in the box without an applied electric field and reduced number density highlighted in the upper left quadrant. Without any force acting on the ions, the ions position is stationary throughout the duration of the simulation.


\subsection{(top-right) constant electric field and reduced number density}
When ions are confined in the box with an applied electric field highlighted in the upper right quadrant, ions accelerate in the direction of the electric field. The potential energy $(E_{p})$ of the positively charged ions in a constant electric field is equivalent to $E_{p}=qU$ where $q$ is the charge of the ion and $U$ is the voltage difference or electric potential difference. The ions accelerate to the right side wall where the potential energy is converted to kinetic energy ($E_k=\frac{1}{2}m{v_d}^2$) where $v_d$ is the velocity of the ion and $m$ is the mass of the ion. In this example, $v_d$ for each ion is increasing until the ions collide on the right side wall of the box. This simulation is analogous to time-of-flight mass spectrometry where the arrival time of ions is dependent upon the mass, and charge, and applied electric field; the collisions with neutrals in high vacuum conditions are rare and often ignored.


\subsection{(bottom-left) no electric field and higher number density}
The kinetic energy of the neutral buffer gas is transferred to the ions where conservation of momentum is maintained. The motion of the ions can be described as Brownian motion.

\subsection{(bottom-right) constant electric field and higher number density}
The simulation on the bottom-right best models that of the IMS experiment. Ions are accelerated by the electric field but the motion is retarded by the presence of neutrals.  Although the instantaneous drift velocity of individual ion is continually changing , the ions are considered to reach an average terminal velocity $(v_d)$ that is dependent on the interactions between the ions and buffer gas in addition to the influence of the electric field and ion. The drift velocity of an ion population ($v_d$) is related to the mobility of the same ion population $(K)$ under a constant electric field $(E)$ from the equation $v_d=KE$.


\section{The Key IMS Scaling Factor $(E/N)$}
A key IMS scaling factor is reduced field strength $(E/N)$ and is an important consideration when reporting mobility measurements. $E/N$ is commonly reported in literature as Townsends (Td), where $1 \;\text{Td} = 10^{-17}\;\text{V}\; \text{cm}^2$.
When $E$ and $N$ are both multiplied by some constant $(\delta)$, the drift velocity of the ion population $(v_d)$ is unaffected (ignoring three-body collisions). Under low-field conditions, the mobility of an ion population ($K$) is \textit{nearly} invariant to $E/N$.

The bottom-right quadrant of the animation shown previously is now located in the top-left corner of the animation provided below. 
The left and right columns correspond to multiplying $E$ by a constant $\delta=1$ and $\delta=2$ respectively where the ion acceleration is multiplied by the equivalent factor $\delta$. The top and bottom rows in the animation below correspond to multiplying $N$ by the constant $\delta=1$ and $\delta=2$ respectively where multiplying the number density ($N$) by the constant $\delta$ reduces the distance to the next collision by the same factor $\delta$. 




The animation shown above demonstrate the similar arrival time distributions for the top-left and bottom-right quadrant where multiplying both $E$ and $N$ by the constant $\delta$ keeps $E/N$ and $v_d$ constant. 





	
\end{document}