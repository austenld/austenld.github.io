\documentclass[10pt,a4paper,final]{article}
\usepackage[latin1]{inputenc}
\usepackage{amsmath}
\usepackage{amsfonts}
\usepackage{amssymb}
\usepackage{graphicx}
\title{Reduced Field Strength}

%%%%%%%%%%%%%%%%%%%%%
%---
%layout: post
%title: Introduction IMS Theory
%---

\begin{document}
The first part of this post highlights why the drift velocity of an ion population depends on the applied electric field and number density for IMS measurements. The second part examines the key scaling factor known as field strength.

%Include E1N1 here?

\section{Electric Field $(E)$ and Number Density $(N)$}

%Include animation here

The drift velocity of an ion population $(v_d)$ is dependent on the applied electric field $(E)$ and number density $(N)$. 
The electric field $(E)$ for a traditional linear drift tube is constant and the number density $(N)$ of the drift gas is determined (under ideal conditions) by the ideal gas law ($PV=nRT=Nk_bT$).
A key IMS scaling factor is known as reduced field strength $(E/N)$ and is an important consideration when reporting mobility measurements. 

The animations below is split into four quadrants where positively charged ions (red diamonds) and neutrals (blue circles) are confined in a box where only elastic collisions are considered.
The ions are initially placed in the center x-axis without any starting initial velocity. The neutrals are dispersed in the confined box with random starting initial velocities and positions.

%Animation goes here
In the animation above, the first and second column corresponds to a simulation in the presence of no electric field and applied constant electric field (the direction of the field is from left to right ) respectively. Similarly, the first and second rows correspond to where the number density is effectively reduced to zero (mimicking high vacuum conditions) and increased number density respectively.


\subsection{(top-left) no electric field and reduced number density}
First, we can consider the simple example when ions are confined in a box without an applied electric field and reduced number density highlighted in the upper left quadrant. Without any force acting on the ions, the ions position is stationary throughout the duration of the simulation.


\subsection{(top-right) constant electric field and reduced number density}
When ions are confined in a box with an applied electric field and reduced number density highlighted in the upper right quadrant, ions accelerate in the direction of the electric field. The potential energy $(E_{p})$ of the positively charged ions in a constant electric field is equivalent to $E_{p}=qU$ where $q$ is the charge of the ions and $U$ is the voltage drop or electric potential difference. The ions accelerate to the right side wall where the potential energy is converted to kinetic energy ($E_k=\frac{1}{2}m{v_d}^2$) where $v_d$ is the velocity of the ion and $m$ is the mass of the ion.
This simulation is analogous to time-of-flight mass spectrometry.


\subsection{(bottom-left) no electric field and higher number density}
The kinetic energy of the neutral buffer gas is transferred to the ions where conservation of momentum is maintained. The motion of the ions is described by Brownian motion.

\subsection{(bottom-right) constant electric field and higher number density}
The simulation on the bottom-right best models that of the IMS experiment. Ions are accelerated by the electric field but the motion is retarded by the presence of neutrals.  Although the instantaneous drift velocity of individual ion is continually changing , the ions are considered to reach a terminal velocity that is dependent on the mobility of the ion population $(K)$.




\section{The Key IMS Scaling Factor $(E/N)$}

\subsection{Double Electric Field}
The ions travel with a higher terminal velocity $(v_d)$ when the electric field is doubled proportional to the defined mobility $(K)$ of the ion population.


\subsection{Double Electric Field and Number Density}
The reduced field strength remains constant when the number density and electric field are doubled. The animation on the right highlights that the terminal velocity $(v_d)$ of the ion population is similar to the animation in the left.
	%When the number density is doubled without changing the electric field setting from previous, the ions travel slower
	%<video width="750" height="375" controls="controls" align="center">)
	%<source src="/animations/IMS_Theory/Efield1xN2x.mp4" type="video/mp4">)
%	</video>
	
%	The electric field setting is now doubled.
%	<video width="750" height="375" controls="controls" align="center">)
%	<source src="/animations/IMS_Theory/Efield2xN2x.mp4" type="video/mp4">)
%	</video>
	
	
	
\end{document}