\documentclass[10pt,a4paper,final]{article}
\usepackage[latin1]{inputenc}
\usepackage{amsmath}
\usepackage{amsfonts}
\usepackage{amssymb}
\usepackage{graphicx}

%%%%%%%%%%%%%%%%%%%%%
%---
%layout: post
%title: Introduction IMS Theory
%---

\begin{document}
\section{Key IMS Scaling Factor ($E/N$- Reduced Field Strength)}
$E/N$ is known as the reduced field strength. Experimentally, $K$ is understood to be constant at low reduced field strengths. We shall now see why this parameter is useful by observing from this simulation the effect of varying the reduced field strength $(E/N)$.
\subsection{No Electric Field ($E/N=0$)}
First, we can consider the effect when ions are confined in a box in the presence of neutrals without the presence of an electric field.
%<video width="750" height="375" controls="controls">)
%<source src="/animations/IMS_Theory/NoEfield.mp4" type="video/mp4">)
%</video>
The red squares represent the ions and the blue dots correspond to the neutrals present. Without the presence of an electric field, the ions travel in a pattern of motion known as Brownian motion. The ions behave similarly to that of the neutral gasses.

\subsection{Applied Constant Electric Field}
Now the motion of the ions change when a constant electric field is applied. The voltage decreases linearly in the direction from left to right.
	%<video width="750" height="375" controls="controls" align="center">)
	%<source src="/animations/IMS_Theory/Efield1xN1x.mp4" type="video/mp4">)
	%</video>

\subsection{Double Electric Field}
The ions travel with a higher terminal velocity $(v_d)$ when the electric field is doubled proportional to the defined mobility $(K)$ of the ion population.


\subsection{Double Electric Field and Number Density}
The reduced field strength remains constant when the number density and electric field are doubled. The animation on the right highlights that the terminal velocity $(v_d)$ of the ion population is similar to the animation in the left.
	%When the number density is doubled without changing the electric field setting from previous, the ions travel slower
	%<video width="750" height="375" controls="controls" align="center">)
	%<source src="/animations/IMS_Theory/Efield1xN2x.mp4" type="video/mp4">)
%	</video>
	
%	The electric field setting is now doubled.
%	<video width="750" height="375" controls="controls" align="center">)
%	<source src="/animations/IMS_Theory/Efield2xN2x.mp4" type="video/mp4">)
%	</video>
	
	
	
\end{document}